\documentclass[11pt]{article}
\usepackage[a4paper,margin=2.8cm]{geometry}
\usepackage{amsfonts}
\usepackage{mathtools}
\usepackage{cancel}
\usepackage{amsthm}
\usepackage{amsmath}
\usepackage{amssymb}
\usepackage{dsfont}
\usepackage[dvipsnames,usenames]{color}
\usepackage[colorlinks=true,linkcolor=NavyBlue,citecolor=NavyBlue,urlcolor=NavyBlue]{hyperref} % for hyperlinks 

% makros 
%%%%%%%%%%%%%%%%%%%%%%%%%%%%%%%%%%%%%%%%%%%%%%%%%%%%%%%%%%%%%%%%%%%%%%%%%%%%
\newcommand{\R}{\mathds {R}}
\newcommand{\red}[1]{\textcolor{red}{#1}} % for color for remarks


\title{Cournot Competition with Idiosyncratic Taxation}

\author{Nate Dwyer and Sandro Lera}

\begin{document}

\maketitle
\setcounter{section}{0}
\section{Model Definition}
We simulate an economy of buyers and sellers as an extension of the Cournot
competition model, with additional buyer-seller specific (`idiosyncratic')
transaction cost (`tax').  The goal of this tax is to avoid market dominance by
one single seller based on initially only small differences in quality.
Instead, the model is aimed at striking a `fair' balance between market
competition and seller protection. 

The `geography' of our model is one-dimensional with periodic boundaries, i.e. a
topological circle, of length $L$.  Due to the periodic boundary conditions, we
measure physical distances with the metric $\min(|x-y|, L-|y-x|)$ for any two
points (`agents') $x$,$y$ on $L$.  Without loss of generality, we set $L=1$.
Our model does not rely on the implemented geography, and any other, potentially
more realistic topology could be considered. 

\subsection{Buyers}
On that circle we place some $|B|$ buyers randomly distributed. The buyers all
have an identical demand curve $P + Q = a$, and each buyer $b\in B$ buys a
quantity $\vec{b}$, where $b_s$ is the amount $b$ buys from the seller $s$. Each
buyer tries to buy as much total quantity as they can while keeping $P + Q \le
a$ i.e.  they try to maximize $\sum_{i=0}^{|S|}b_s$. 

\subsection{Sellers}
On that circle we will also find some $|S|$ sellers, who all offer the same
product, but at potentially different prices and in potentially different
quantities. We denote the prices that are offered by a vector $\vec{p}$ and the
quantities by the vector $\vec{q}$, where the seller $s$ offers $q_s$ amount of
the product at a price $p_s$. Each seller chooses $q_s$ and $p_s$ simultaneously
in an attempt to maximize profit.  

\subsection{Idiosyncratic tax}
For any buyer $b$ and seller $s$, we define a product-distance $D = D(b,s)$ such
that the seller s that is physically closest to b has distance $D(b,s)=1$, the
next closest has distance $D=2$, and so on. 

If buyer $b$ buys from seller $s$, the total price, including transaction cost
(tax), is given by $p(b,s) = p_s  \cdot D(b,s)^\gamma$ where $\gamma$ is the
scaling coefficient of the tax. 

\subsection{Profit Function}

The profit function for seller $s$ reads
\begin{equation}
    \Pi_s = p_sq_s^{sold} - cq_s, 
    \label{eq:profit_function}
\end{equation}
where $c$ is the price to produce each unit of product, and $q_s^{sold}$ is the
quantity the seller $s$ sells (which may be less but never more than the
quantity $q_s$ that was produced).  Below, we then introduce the profit vector
$\vec{\pi} = (\pi_1, \ldots, \pi_{|S|})$ to find the Nash-equilibrium based on
the profit functions of all sellers.


\section{Mathematical Representation}
We construct a set of buyers $B$ and a set of sellers $S$, and transaction cost
function $t: (B\times S)\rightarrow \R$, which has subfunctions $t_b:
S\rightarrow \R$ s.t. for all $s\in S$, $t(b,s) = t_b(s)$.
\red{ I don't see what are these sub-functions for.  The way I see it, a
specific customer-seller tax is a mapping from $B \times S$ to $\R$.  There is
no subfunction. NATE Response: $t_b$ is used in the Algorithm part.}

Also, we construct an demand curve for all $b\in B$, namely $P = a - Q$.

\section{Function to put into Gambit}

We construct a function $\Pi : (\R\times\R)^{|S|} \rightarrow \R^{|S|}$ s.t.
$\Pi(\vec{s}) = \vec{\pi}$, where the elements of $\vec{s}$ are the tuples
$(q_s,p_s)$ listing the quantity produced and the price offered by the seller
$s$, and the elements of $\vec{\pi}$ are $\pi_s$, the profits of the seller $s$.

Given $\vec{s}$, $\vec{\pi}$ is calculated by first calculating
$\vec{q}^{sold}$, the vector of total quantity sold, then using it to calculate
$\vec{\pi}$ using equation \eqref{eq:profit_function}.  

Calculate $\vec{q}^{sold} = \sum _{b\in B} \vec{b}$, where $\vec{b}$ represents
the amount buyer $b$ buys from each seller. 
\red{You have two matrices, $Q$ and $Q^{sold}$, and the same for prices. Then
you have $Q_{sb}$ instead of $Q_{ji}$. NATE Response: currently, I think it
might be clearer with vectors.  If we do go to matrices, then we should probably
go there for price as well, and we can get rid of the subfunctions that way.}


\subsection{Algorithm to calculate $\vec{q}^{sold}$}

To determine $\vec{b}$, do the following:

For each $b_s \in \vec{b}$, have it maximize the total quantity bought by $b$
while remaining under the individual demand curve $P=a-Q$.  The resulting vector
that lists the amount $b$ buys from each seller is $\vec{b}$.  Subtract
$\vec{b}$ from $\vec{q}^{unsold}$ to get a new $\vec{q}^{unsold}$.


\begin{enumerate}
    \item Set $\vec{q}^{unsold} = \vec{q}$ to start.
    \item For each $b\in B$:
	\item Calculate $b$'s perceived prices: $t_b(s_s) * p_s = \vec{p}^b$.  
    \item Turn $\vec{p}^b$ into a $|S|$ by 2 matrix $M$ with the first column
        being the price, $\vec{p}$ the second column being the remaining
        quantity, $\vec{q}^{unsold}$. 
    \item Sort $M$'s rows by the perceived price column, and re-index the
        rows $0, 1, \cdots , i, \cdots , |S|$. 
    \item Find the maximal index $i$ and quantity $q^{*}_i \le q^{unsold}_i$
        such that $p_i + q^{*}_i + \sum_{k=0}^i q_k \le a$. 
    \item Now we have a vector $\vec{b}'$ of amount bought by $b$ : $\vec{b}' =
        <q_0, q_1,\cdots q_{j-1}, q^{*}_i, 0, 0, \cdots, 0>.$
    \item Re-index this new vector using the seller order instead of the price
        index to get $\vec{b}$.
    \item Set $\vec{q}^{unsold} = \vec{q}^{unsold}-\vec{b}$.
    \item Go to step 3.
    \item Calculate $\vec{q}^{sold} = \sum _{b\in B} \vec{b}$.
    \item Calculate $\vec{\Pi} = \vec{p}\vec{q}^{sold} - c\vec{q}$, 
\end{enumerate}

\section{Open Questions}

\begin{itemize}

    \item With this model, we want to demonstrate that initially small
        differences in quality do not lead to proportional feedback mechanisms
        that lead to an unequal output distribution So it might be questionable
        if the current Cournot model is the model of our interests, because it
        assumes all buyers set the same price. 

\end{itemize}

\end{document}
