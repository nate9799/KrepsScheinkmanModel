\documentclass[14pt]{article}
\usepackage{amsfonts}

\usepackage{mathtools}
\usepackage{cancel}
\usepackage{amsthm}
\usepackage{amsmath}
\usepackage{amssymb}

\usepackage{dsfont}
\newcommand{\R}{\mathds {R}}
\title{ABM of Cournot Competition with Idiosyncratic Tax}
\author{Nate Dwyer and Sandro Lera}

\begin{document}

\maketitle
\setcounter{section}{0}
\section{Model Definition}
We simulate a basic economy of buyers and sellers by basing it on the
Cournot competition model. 

The 'geography' of our model is one-dimensional with periodic boundaries, 
i.e. a topological circle. We denote the length of that circle by $L$. We 
measure physical distances with the metric $min(|x-y|, L-|y-x|)$ for 
any two points ('agents') $x$,$y$ on $L$. Without loss of generality,
we set $L=1$. 

\subsection{Buyers}
On that circle we place some $|B|$ buyers randomly distributed. The buyers
all have an identical demand curve $P=a-Q$, and each buyer $b\in B$ buys a 
quantity $\vec{b}$, where $b_i$ is the amount $b$ buys from the $i$th
seller. Each buyer tries to buy as much total quantity as they can while
staying under the curve, i.e. they try to maximize $sum(\vec{b})$. 

\subsection{Sellers}
On that circle we will also find some $|S|$ sellers, who all offer the same
product, but at potentially different prices and in potentially different
quantities. We denote the prices that are offered by a vector $\vec{p}$ and 
the quantities by the vector $\vec{q}$, where the $i$th seller offers $q_i$
amount of the product at a price $p_i$. Each seller chooses $q_i$ and $p_i$ 
simultaneously in an attempt to maximize profit.  

\subsection{Idiosyncratic tax}
For any buyer $b$ and seller $s$, we define a product-distance $D = D(b,s)$ 
such that the seller s that is physically closest to b has distance 
$D(b,s)=1$, the next closest has distance $D=2$, and so on. 

If buyer b buys from seller s, the total price, including transaction cost
(tax), is given by $p(b,s) = p_s *  D(b,s)^\gamma$ where gamma is a 
scaling coefficient. 

\subsection{Profit Function}
The profit function is as follows: $\Pi_i = p_i * q_i^s - c*q_i$, where $c$
is the price to produce each unit of product, and $q_i^s$ is the quantity 
the $i$th seller sells.  


\section{Setup}
We construct a set of buyers $B$ and a set of sellers $S$, and transaction cost
function $t: (B\times S)\rightarrow \R$, which has subfunctions $t_b:
S\rightarrow \R$ s.t. for all $s\in S$, $t(b,s) = t_b(s)$.

Also, we construct an demand curve for all $b\in B$, namely $P = a - Q$.

\section{Function to put into nashpy}
We construct a function $\Pi : (\R\times\R)^{|S|} \rightarrow \R^{|S|}$ s.t.
$\Pi(\vec{s}) = \vec{\pi}$, where the elements of $\vec{s}$ are the tuples
$(q_i,p_i)$ listing the quantity produced and the price offered by the $i$th
seller, and the elements of $\vec{\pi}$ are $\pi_i$, the profits of the $i$th seller.

Given $\vec{s}$, $\vec{\pi}$ is calculated by first calculating $\vec{q}^s$, the
vector of total quantity sold, then using it to calculate $\vec{\pi}$ using the formula $\pi_i = p_iq^s_i$.  

$q^s_i = \sum _{b\in B} b_i$, where $b_i$ is the amount seller $b$ buys from
the $i$th seller.  

Each buyer $b$ chooses a vector $\vec{b}$, where each element of the vector
$b_i$ is the amount $b$ bought from the $i$th buyer.  

Set $\vec{q}^{unsold} = \vec{q}$

For each $b_i \in \vec{b}$, have it maximize the total quantity bought by $b$
while remaining under the individual demand curve $P=a-Q$.  The resulting
vector that lists the amount $b$ buys from each seller is $\vec{b}$.  Subtract $\vec{b}$ from $\vec{q}^{unsold}$ to get a new $\vec{q}^{unsold}$.

\subsection{Algorithm to determining what a buyer purchases}
To determine $\vec{b}$, do the following:

Calculate $b$'s perceived prices, $t_b(s_i) * p_i = \vec{p}^b$, then turn it
into a $|S|$ by 2 matrix $M$ with the first column being the price, $\vec{p}$
the second column being the remaining quantity, $\vec{q}^{unsold}$. 

Sort $M$'s rows by the perceived price column, and re-index the rows\newline
$0, 1, \cdots , j, \cdots , |S|$. 

Find the maximal index $j$ and quantity $q^{*}_j \le q^{unsold}_j$ such that
$p_j + q^{*}_j + \sum_{k=0}^j q_k \le a$.  Now we have a vector of amount
bought by $b$ as follows: $<q_0, q_1,\cdots q_{j-1}, q^{*}_j, 0, 0, \cdots, 0>.$

Finally, re-index this new vector using the $i$-index instead of the $j$ index
to get $\vec{b}$.
\end{document}
